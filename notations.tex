\addcontentsline{toc}{chapter}{Notations}
\chapter*{Notations}

\section*{References}

When available the bibliography style features three different links associated with three different colors: links to the journal/editor website or to a numerical version of the paper are in \textcolor{LinkJournal}{red}, links to the ADS website are in \textcolor{LinkADS}{blue} and links to the arXiv website are in \textcolor{LinkArXiv}{green}.

\section*{Results}
\begin{result}
    The important passages of the thesis are highlighted in grey boxes.
\end{result}

\begin{tcresult*}{Name}
    If the result is named (e.g. equations of $\cdots$) it will be displayed like so instead.
\end{tcresult*}


\section*{Warning}
\begin{marker}
    Important caveats or warning of the thesis will be highlighted like this.
\end{marker}

\section*{Definitions}
\begin{tcdefinition}{style of definition}{}
Definition are always numbered and named like so
\end{tcdefinition}

\section*{Theorems}
\begin{tctheorem}{style of theorems}{}
Theorems are always numbered and named like so
\end{tctheorem}

\section*{Acronyms}

\begin{tabular}{ll}
    SI		& Streaming-instability \\
    M-SI	& magnetised Streaming-instability\\
    NIM-SI	& non-ideal magnetised Streaming-instability\\
\end{tabular}

\section*{Mathematical conventions}
\label{sec::Notation}


We denote indices running from 0 to 3 by Greek letters ($\alpha$, $\beta$, $\gamma$, ...) and indices running from 1 to 3 by Roman letters ($a$, $b$, $c$, ...).

\section*{Variables}

\begin{tabular}{ll}
    $\T g$				& Lorentzian metric (-+++) \\
    $\T\nabla$			& Spacetime connection (Lorentzian or Galilean)
\end{tabular}
